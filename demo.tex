\documentclass[10pt]{beamer}
\usetheme[progressbar=frametitle]{metropolis}
\usepackage{appendixnumberbeamer}

\usepackage{amsmath}
\usepackage{mathrsfs}
\usepackage{amsfonts}
\usepackage{amsopn}
\usepackage{amstext}
\usepackage{amssymb}
\usepackage{amscd}
\usepackage{lmodern}

\usepackage{booktabs}
\usepackage[scale=2]{ccicons}

\usepackage{pgfplots}
\usepgfplotslibrary{dateplot}

\usepackage{xspace}
\usepackage{changes}   % fuer durchgestrichene Texte
\newcommand{\themename}{\textbf{\textsc{metropolis}}\xspace}

\newcommand{\R}{\mathbb R}
\providecommand{\abs}[1]{\lvert#1\rvert}
\providecommand{\norm}[1]{\lVert#1\rVert}

\title{Moderne Portfoliostrategien für Mandatslösungen}
\subtitle{}
\date{\today}
\date{}
\author{Thomas Schmelzer\\ thomas.schmelzer@lobnek.com}
\institute{Lobnek Wealth Management}
% \titlegraphic{\hfill\includegraphics[height=1.5cm]{logo.pdf}}

\begin{document}

\maketitle

%\begin{frame}{Table of contents}
%  \setbeamertemplate{section in toc}[sections numbered]
%  \tableofcontents[hideallsubsections]
%\end{frame}

\section{Einführung}

\begin{frame}{Modern? Ein Werkzeugkasten für Quants}

\begin{itemize}
\item Quantitativ oder neu \textit{Quantamental}?
\item Bayes'sche Statistik (Bayes, 1763 (posthum) und z.B. Black und Litterman, 1992)
\item Regression, Methode der kleinsten Quadrate (Gauss, 1795, 18jährig)
\item Lasso (Tibishirani, 1996) und LARS (Efron et al, 2004).
\item SOCP (Nesterov et al, 1994) und Semidefinite Programming.
\end{itemize}

Ist diskretionär auch modern? Ist es eine Strategie?
%\href{https://arxiv.org/abs/1310.3396}{\textsc{Seven in Sins in Portfolio Optimization}} by Schmelzer and Hauser (2013).
\end{frame}

\begin{frame}[fragile]{Ziele}
\begin{itemize}
\item \deleted{Obfuskation und Rechtbeistand}
\item Maximaler Profit für den Kunden durch Allokation in liquide Instrumente unter Beachtung verschiedenster Nebenbedingungen.
\begin{itemize}
\item Andere Zielfunktionen sind möglich.
\item Das ist ein Optimierungsproblem in mindestens $n$ Dimensionen (Anzahl der Assets).
\item Marked to Market. Sonst Schindluder? 
\end{itemize}
\end{itemize}
\end{frame}


\begin{frame}{Die $5$ Portfoliomanager-Typen}
      %\metroset{block=fill}
      \begin{alertblock}{Diskretionär}
        Keine Anwendung quantitativer Strategien.
      \end{alertblock}
      
      \begin{alertblock}{Experimentell}
       Quantitative Strategien probiert, aber verworfen.
      \end{alertblock}
      
      \begin{alertblock}{Musisch}
        Quantitative Strategien anwendend, aber Ergebnisse nur als Startpunkt.
      \end{alertblock}
      
      \begin{alertblock}{Modern}
        Quantitative Strategien anwendend und den Ergebnissen vertrauend.
      \end{alertblock}
      
      \begin{alertblock}{Postmodern}
        Völlig losgelöst von allen Strategien. Quantitativ als reines Marketingargument.
      \end{alertblock}
\end{frame}


\section{Gezähmte Portfolios}

\begin{frame}{Das $1/n$ Portfolio}
Jedes Portfolio soll als Lösung eines Optimierungsproblems dargestellt werden:
\begin{align}
w^*=\arg\min_{w\in\R^n}&\,\sqrt{w_1^2  + \ldots w_n^2} = \norm{w}_2 \\
\text{s.t. }&\sum w_i = 1, w\geq 0\nonumber
\end{align}
\begin{itemize}
\item Für die Geometer: Finde die kleinste Kugel, die den Einheitssimplex im positiven Quadranten berührt (im Raum der Gewichte).
\item Ein Optimierungsproblem für das $1/n$ Portfolio ist die Kanone für Spatzen? 
\end{itemize}
\end{frame}

\begin{frame}{Das Minimum-Varianz Portfolio}
Hier gilt
\begin{align}
w^*=\arg\min_{w\in\R^n}&\, \norm{Rw}_2 \\
\text{s.t. }&\sum w_i = 1, w\geq 0\nonumber
\end{align}

\begin{itemize}
\item $Rw = R_1 w_1 + R_2 w_2 + \ldots$ ist der Return-Vektor des Portfolios. $R_i$ ist der Return-Vektor des $i$ten Assets.
\item Für die Geometer: Finde das kleinste Ellipsoid (induziert durch die Matrix $R^T R$), das den Einheitssimplex im positiven Quadranten berührt (im Raum der Gewichte).
\item Lösungen oft sehr dünnbesetzt und recht instabil (abhaengig von der Laenge der Geschichte).
\end{itemize}
\end{frame}


\begin{frame}{Das gezähmte Minimum-Varianz Portfolio}
\begin{align}
w^*=\arg\min_{w\in\R^n}&\, \norm{Rw}_2 + \lambda \norm{w}_2 \\
\text{s.t. }&\sum w_i = 1, w\geq 0\nonumber
\end{align}
Das Verhalten das Portfolios ist abhängig von der Wahl des freien Parameters $\lambda$:
\begin{itemize}
\item $\lambda >> 0$. Zweiter Term dominiert. Nähe zum $1/n$ Portfolio.
\item $\lambda \approx 0$. Erster Term dominiert. Nähe zum Minimum-Varianz Portfolio.
\item Für die Geometer: Wir ziehen das Ellipsoid hin zu einer Kugel.
\end{itemize}
Wir interpretieren $\lambda$ als Gaspedal. 
\end{frame}

\begin{frame}{Eine neue Dimension}
\begin{itemize}
\item Das Minimum-Varianz Portfolio ist ein Eckpfeiler (für $\lambda = 0$) eines allgemeineren Portfolios.
\item Wir haben dem Problem eine weitere Dimension gegeben und können darin die Balance des Portfolios steuern. 
\item Die Störung mit der $2$-Norm der Gewichte ist auch bekannt als Ridge Regression (Statistik) oder Tikhonov Regularization (Mathematik).
\end{itemize}
\end{frame}

\section{Schlanke Portfolios}
\begin{frame}{Private Banking für UHNWI}
\begin{alertblock}{Positionswechsel? Aus der Praxis}
Ein Kunde $X$ hat ca. $40$ Positionen in seinem Portfolio. Einmal im Monat findet ein Rebalancing statt. Der Kunde wird in einem  Gespräch (ca. $30$ Minuten)  über alle Positionswechsel vorab informiert. Wir wollen deshalb Positionswechsel vermeiden!
\end{alertblock}

Das ist ein klassisches \alert{Selektionsproblem}. Reiche Literatur. Keine Notwendigkeit hier das Rad neu zu erfinden. 
\end{frame}

\begin{frame}{Gleiches Problem}
\begin{align}
w^*=\arg\min_{w\in\R^n}&\, \norm{Rw}_2 + \lambda \norm{w-w^0}_1 \\
\text{s.t. }&\sum w_i = 1, w\geq 0\nonumber
\end{align}
Wir verwenden jetzt eine alternative Norm im zweiten Term. Es gilt:
\[ 
\norm{w - w^0}_1 = \sum_{i=1}^n \abs{w_i-w^0_i}
\]
\begin{itemize}
\item $\lambda >> 0$. Zweiter Term dominiert. Keine Trades
\item $\lambda \approx 0$. Erster Term dominiert. Nähe zum Minimum-Varianz Portfolio.
\end{itemize}
\end{frame}

\begin{frame}{Eine neue Dimension}
\begin{itemize}
\item Das Minimum-Varianz Portfolio ist ein Eckpfeiler (für $\lambda = 0$) eines allgemeineren Portfolios.
\item Wir haben dem Problem eine weitere Dimension gegeben und können darin die Anzahl der Updates des Portfolios steuern. 
\item Die Störung mit der $1$-Norm der Gewichte ist auch bekannt als Lasso Regression (Statistik). 
\end{itemize}
\end{frame}

\section{Gezähmte und schlanke Portfolios}
\begin{frame}{Das Elastic Net}
\begin{align}
w^*=\arg\min_{w\in\R^n}&\, \norm{Rw}_2 + \lambda_1 \norm{w-w^0}_1 + \lambda_2 \norm{w}_2 \\
\text{s.t. }&\sum w_i = 1, w\geq 0\nonumber
\end{align}
Das Minimum-Varianz Portfolio wurde hier um $2$ Dimensionen erweitert
\begin{itemize}
\item $\lambda_1$ reguliert die Anzahl der Trades.
\item $\lambda_2$ reguliert die Balance im Portfolio.
\end{itemize}
Diese Art der Regression wird als Elastic Net (Statistik) bezeichnet.
\end{frame}

\section{Ergebnisse am SMI}

\begin{frame}

\end{frame}

\section{Zusammenfassung}

\begin{frame}
\begin{itemize}
\item Wir wenden Standardwerkzeuge aus der Statistik an um die Balance und die Aktivität eines Portfolios zu steuern. Keine Hacks.
\item Erst die Einbettung des Allokationsproblems in die Sprache der konvexen Optimierung erlaubt uns diesen Zugang. Gleichzeitig können wir weitere Rahmenbedingungen einführen (z.B. Maximalgewichte für Gruppen von Assets).
\item Leider noch viele weitere Hürden für die Anwender: \href{https://arxiv.org/abs/1310.3396}{\textsc{Seven in Sins in Portfolio Optimization}} by Schmelzer and Hauser (2013). Gibt es eine Hemmschwelle in der Praxis?
\end{itemize}
\end{frame}

{\setbeamercolor{palette primary}{fg=black, bg=yellow}
\begin{frame}[standout]
  Fragen?
\end{frame}
}

\end{document}
